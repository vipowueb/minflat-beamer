\documentclass[xetex,notheorems,hyperref={pdfpagelabels=true},xcolor={table}]{beamer}
\usetheme{minflat}

\usepackage{booktabs}
\usepackage[scale=2]{ccicons}

\usepackage[style=american]{csquotes}

\usetikzlibrary{decorations.pathreplacing, decorations.pathmorphing,calc,arrows,positioning}

%%% enable notes on second screen
%\usepackage{pgfpages}
%\setbeameroption{show notes on second screen=right}
%\setbeamertemplate{note page}[compress]

	
%%%%%%%%%%%%%%%%%%%%%%%%%%%%%%%%%%%%%%%%%%%%%%%%%%%
%%%%	define content of title page
%%%%%%%%%%%%%%%%%%%%%%%%%%%%%%%%%%%%%%%%%%%%%%%%%%%
\def\talkTitle{minflat Beamer Theme}
\def\talkShortTitle{Short title}
\def\talkSubtitle{Your subtitle (Talk @ etc.) }
\def\talkKeywords{Keywords}

%% Define meta data of pdf
\hypersetup{
    pdftitle={Slides to talk - \talkTitle},
	pdfsubject={\talkTitle},
	pdfauthor={Robert Baumgarth},
	pdfkeywords={\talkKeywords},
	colorlinks=false
}


%%%%%%%%%%%%%%%%%%%%%%%%%%%%%%%%%%%%%%%%%%%%%%%%%%%
%%%%	set content of title page
%%%%%%%%%%%%%%%%%%%%%%%%%%%%%%%%%%%%%%%%%%%%%%%%%%%
\title[\talkShortTitle]{\talkTitle}  
\subtitle{\talkSubtitle} 
\author{Robert Baumgarth -- robert.baumgarth@uni.lu}
\DTMlangsetup[en-GB]{ord=raise,monthyearsep={,\space}}
\date{\DTMtoday}
\institute{%
	\includegraphics[width=.5\textwidth]{./gfx/uni_lu_logo_gr.jpg}%
}

%%%%%%%%%%%%%%%%%%%%%%%%%%%%%%%%%%%%%%%%%%%%%%%%%%%
%%%%	theorem tools, theorem and proof styles
%%%%%%%%%%%%%%%%%%%%%%%%%%%%%%%%%%%%%%%%%%%%%%%%%%%
\setbeamertemplate{theorem}[ams style]
%\setbeamertemplate{theorems}[numbered]

\newcounter{chapter}
\setcounter{chapter}{1}
\theoremstyle{plain}
\newtheorem{theorem}{Theorem}[section]
\newtheorem{lemma}[theorem]{Lemma}
\newtheorem{proposition}[theorem]{Proposition}
\newtheorem{corollary}[theorem]{Corollary}

\theoremstyle{definition}
\newtheorem{conclusion}[theorem]{Conclusion}
\newtheorem*{definition}{Definition}
\newtheorem*{remark}{Remark}
\newtheorem*{dummyblock}{dummyblock}

\theoremstyle{example}
\newtheorem{example}[theorem]{Example}

\newenvironment<>{dummyblock}[1]{%
	\setbeamercolor{block title}{fg=white,bg=white}%
	\setbeamercolor{block body}{fg=normal text.fg,bg=white}%
	\begin{block}#2{#1}}{\end{block}}


\begin{document}

\renewcommand{\leq}{\leqslant}
\renewcommand{\geq}{\geqslant}

\renewcommand\theta\vartheta


%%%%%%%%%%%%%%%%%%%%%%%%%%%%%%%%%%%%%%%%%%%%%%%%%%%
%%%%	title page
%%%%%%%%%%%%%%%%%%%%%%%%%%%%%%%%%%%%%%%%%%%%%%%%%%%
\begin{frame}[plain]
	\titlepage
\end{frame}


%%%%%%%%%%%%%%%%%%%%%%%%%%%%%%%%%%%%%%%%%%%%%%%%%%%
%%%%	toc
%%%%%%%%%%%%%%%%%%%%%%%%%%%%%%%%%%%%%%%%%%%%%%%%%%%
\begin{frame}
	\tableofcontents
\end{frame}


%%%%%%%%%%%%%%%%%%%%%%%%%%%%%%%%%%%%%%%%%%%%%%%%%%%
%%%%	introduction
%%%%%%%%%%%%%%%%%%%%%%%%%%%%%%%%%%%%%%%%%%%%%%%%%%%
\section{Introduction}
\begin{frame}[fragile]
	\frametitle{minflat}
	The \emph{minflat} theme is a Beamer theme in modern flat design, i.e. emphasising a minimal yet functional design, primarily designed for mathematical talks.\\[.5em]

	You can enable the theme by loading:
	\begin{verbatim}    \documentclass{beamer}
    \usetheme{minflat} \end{verbatim}
    Note that XeTeX and the free Museo Sans 300 font need to be installed. Moreover this theme requires that the following packages are installed:
	\begin{itemize}
		\item \emph{tikz}
		\item \emph{datetime2}
	\end{itemize}
\end{frame}


\begin{frame}{Design}
	You can choose between two color schemes. A {\bfseries\color{darkpurple} blue violet} and a {\bfseries\color[RGB]{137,57,94} red purple} variant. Corresponding colours used are green for the {\color{progressbar green} progress bar} (and {\color{beamergreen} examples}) and orange for {\color{beamerorange} alert elements} (cf. \hyperlink{slide::ex-alert-el}{below}).\\[.5em]
	
	Sections always start showing an overlay slide containing the section name and a nice progress bar.\\[.5em]
	
	All slides have a progress bar on top where the current section is indicated by highlighted green blob and a slightly bolder name.\\[.5em]
	
	The date uses the {\fontsize{9pt}{9pt}\selectfont\texttt{datetime2}} package so you can change the format if you wish. You should replace the logo on the title page with our own by setting the appropriate path or just comment out with no harm. At the moment I am a PhD student (Assistant-doctorant) at the University of Luxembourg. 
\end{frame}

\begin{frame}[fragile]{Theme Options \& Remarks}
	\begin{center}
		\arrayrulecolor{darkpurple}
		\begin{tabular}[]{ll}
			\toprule
			{\bfseries Option} 	& {\bfseries Description}\\
			\midrule
			purple				& switch to {\bfseries\color[RGB]{137,57,94} red purple} colour scheme\\[0.5em]
			xcolor={table}		& should be activated to colour tables\\[0.5em]
			notheorems			& should be activated to use the dummyblock definition\\
			\bottomrule
		\end{tabular}
	\end{center}	
	Note that if you use the
	\begin{verbatim}  \usepackage{pgfpages}
  \setbeameroption{show notes on second screen=right}
  \setbeamertemplate{note page}[compress] \end{verbatim}
    to enable notes on the second screen there is a bug in beamer that normal text on frames becomes white. So I added a \texttt{dummyblock} wrapper to solve this issue.
\end{frame}


%%%%%%%%%%%%%%%%%%%%%%%%%%%%%%%%%%%%%%%%%%%%%%%%%%%
%%%%	Blocks, Alerts \& Math Environments
%%%%%%%%%%%%%%%%%%%%%%%%%%%%%%%%%%%%%%%%%%%%%%%%%%%
\section{Blocks, Alerts \& Math Environments}
\begin{frame}{Blocks, Alerts \& Math Environments}
	\begin{block}{Notation}
		This is some notation.
	\end{block}

	\begin{definition}
		This is a definition.
	\end{definition}
	
	\begin{remark}
		And this is a remark.
	\end{remark}
\end{frame}

\begin{frame}
	\begin{theorem}[Existence \& Uniqueness for ...]
		A theorem is important so it should be emphasised!
	\end{theorem}

	\begin{proposition}
		A proposition may be a little less important but it's also worth emphasising!
	\end{proposition}
	
	In the same way we can create lemmata \& corollaries.
\end{frame}

\begin{frame}\label{slide::ex-alert-el}
	\begin{example}
		Examples should be made. Of course normally only the mathematician understands why this text on the blackboard should be an example.
	\end{example}

	Finally
		
	\begin{alertblock}{Alert, alert}
		An alert block has a catchy colour.
	\end{alertblock}
\end{frame}


{
	\definecolor{darkpurple}{RGB}{137,57,94}
	\definecolor{beamergreen}{RGB}{57,137,100}
	\definecolor{beamerorange}{RGB}{255,175,0}
	\begin{frame}{Red purple colour scheme}
		Finally let us have a look at the second colour scheme:
		
		\begin{theorem}[Existence \& Uniqueness for ...]
			A theorem is important so it should be emphasised!
		\end{theorem}
	
		\begin{example}
			Examples should be made. Of course normally only the mathematician understands why this text on the blackboard should be an example.
		\end{example}
	
		\begin{alertblock}{Alert, alert}
			An alert block has a catchy colour.
		\end{alertblock}
	\end{frame}
}


%%%%%%%%%%%%%%%%%%%%%%%%%%%%%%%%%%%%%%%%%%%%%%%%%%%
%%%%	Overlays \& Images
%%%%%%%%%%%%%%%%%%%%%%%%%%%%%%%%%%%%%%%%%%%%%%%%%%%
\section{Overlays \& Images}
\begin{frame}[fragile]
	\frametitle{Overlays \& Images}
	Complying with the old saying: \enquote{A picture speaks a thousand words}, we can create frames which only contain a picture by
	\begin{center}
		\texttt{\textbackslash imageFrame\{imageURL\}.}
	\end{center}
	We can also define an overlay on the left oder right side of the frame. By default the overlay has a width of 150pt. We can adjust as an optional argument.
	\begin{verbatim} \imageFrameOverlayLeft[optional width]{%
    ./gfx/horizontallift.pdf}{%
    Want big impact?}{%
    Use a big picture.} \end{verbatim}
\end{frame}

%%%%	image frame with overlay left
\imageFrameOverlayLeft{%
./gfx/horizontallift.pdf}{%
Want big impact?}{%
Use a big picture.}

%%%%	image frame with overlay right, custom size
\imageFrameOverlayRight[150pt]{%
./gfx/horizontallift.pdf}{%
Need a bigger overlay on the right side?}{%
}


%%%%%%%%%%%%%%%%%%%%%%%%%%%%%%%%%%%%%%%%%%%%%%%%%%%
%%%%	Listings, Tables, Highlighted Text \& Tikz
%%%%%%%%%%%%%%%%%%%%%%%%%%%%%%%%%%%%%%%%%%%%%%%%%%%
\section{Listings, Tables, Highlighted Text \& Tikz}

\begin{frame}{Listings}
	\begin{itemize}
		\item Item $\sharp$1
		\item Item $\sharp$2
			\begin{itemize}
				\item Subitem 2.$\sharp$1
				\item Subitem 2.$\sharp$2
			\end{itemize} 
	\end{itemize}
	and so on. We can also create highlighted lists:
	\begin{itemize}[<+- | alert@+>]
		\item \alert<4>{\only<-3>{Hi!}\only<4>{or here?}}
		\item you
		\item there!
	\end{itemize}
\end{frame}

\begin{frame}{Tables}
	\begin{center}
		\arrayrulecolor{darkpurple}
		\begin{tabular}[]{lrrl}
			\toprule
								& \multicolumn{1}{c}{{\bfseries Dual space}}
			                    & \multicolumn{1}{c}{{\bfseries Reflexive}}
			                    & \multicolumn{1}{l}{{\bfseries Norm}} \\
			\midrule
			$\mathbb K^n$	& $\mathbb K^n$			& Yes			& $\Vert x\Vert_2 =\left(\sum_{i=1}^n |x_i|^2\right)^{\frac 12}$\\[0.5em]
			$\ell_p$			& $\ell_q$				& Yes			& $\Vert x\Vert_p = \left( \sum_{i=1}^\infty |x_i|^p \right)^{\frac 1p}$\\[0.5em]
			$\ell_1$			& $\ell_\infty$			& \alert{No}	& $\Vert x\Vert_1 = \sum_{i=1}^\infty |x_i|$\\[0.5em]
			$\ell_\infty$	& {\small complicated}	& \alert{No}	& $\Vert x\Vert_\infty = \sup_i |x_i|$\\[0.5em]
			$L^p(\mu)$		& $L^q(\mu)$			& Yes			& $\Vert f\Vert_p = \left( \int |f|^p \, \mathrm d\mu \right)^{\frac 1p}$\\[0.5em]
			$L^1(\mu)$		& $L^\infty(\mu)$ 		& \alert{No} 	& $\Vert f\Vert_1 = \int |f| \, \mathrm d\mu$\\
			\bottomrule
		\end{tabular}
	\end{center}
\end{frame}

%%%%%%%%%%%%%%%%%%%%%%%%%%%%%%%%%%%%%%%%%%%%%%%%%%%
%%%%	highlighted frame with number and 
%%%%	optional subtext
%%%%%%%%%%%%%%%%%%%%%%%%%%%%%%%%%%%%%%%%%%%%%%%%%%%
\highlightedFrame[such an impressive number should be big!]{89.432.567}

\begin{frame}{Tikz}
	\begin{center}
	    \begin{tikzpicture}[scale=.8]
	        \pgfmathsetseed{2236}
	
	        \fill (0,0) circle (2pt);
		    \draw (0,0) ellipse (5 and 3);
	    		\node at (5,-2.5) {$D \subsetneq \mathbb R^n$};
	        
	        \draw[decorate, decoration={random steps,segment length=5pt,amplitude=10pt}] [darkpurple] plot [smooth, tension=1] coordinates { (0,0) (2,0) (0,2.5) (-3,0) (0,-1.5) (4,1.8)};
	        \node[darkpurple] at (2.8,-1.5) {BM with $p_t$, $T_t$};
	        
	        \node at (5.2,1.8) {$\partial D \ni x_0$};
	        \fill (4,1.8) circle (1pt);
	        
	        \draw[->,beamergreen,thick] (4,1.8) -- (3,1);
	        
	        \draw[->,beamerorange,thick] (4,1.8) -- (5,2.6);
	        \fill (4,1.8) circle (.5pt);
	    \end{tikzpicture}
	\end{center}
	
	\begin{itemize}
	    \item Trap, $x_0$ absorbing = Killing Brownian motion = Dirichlet problem
	    \item {\color{beamergreen} Reflected Brownian motion = Neumann problem}
	    \item {\color{beamerorange} Wait an go on = Sticky Brownian motion}
	\end{itemize}
\end{frame}

%%%%%%%%%%%%%%%%%%%%%%%%%%%%%%%%%%%%%%%%%%%%%%%%%%%
%%%%	thanks slide, not in toc
%%%%%%%%%%%%%%%%%%%%%%%%%%%%%%%%%%%%%%%%%%%%%%%%%%%
\sectionNotInTocAndNavigation{Thank you. Questions?}


%%%%%%%%%%%%%%%%%%%%%%%%%%%%%%%%%%%%%%%%%%%%%%%%%%%
%%%%	conclusion frame
%%%%%%%%%%%%%%%%%%%%%%%%%%%%%%%%%%%%%%%%%%%%%%%%%%%
\begin{frame}{Conclusion}

  Let me close addressing some thanks for the inspiration to the \href{https://github.com/matze/mtheme/}{metropolis mtheme}. This theme and its sources \& demo are completely hosted by

  \begin{center}\url{https://github.com/vipowueb/minflat-beamer}\end{center}

  It \emph{itself} is licensed under the
  \href{http://creativecommons.org/licenses/by-sa/4.0/}{Creative Commons
  Attribution-ShareAlike 4.0 International License}.

  \begin{center}\ccbysa\end{center}
\end{frame}

\end{document}